
%\documentclass[12pt,a4paper]{article}
\documentclass[letterpaper,12pt,titlepage]{article}

\usepackage{hyperref}
\usepackage{listings}


\hypersetup{
	colorlinks,
	citecolor=black,
	filecolor=black,
	linkcolor=black,
	urlcolor=black
}

% This is a comment
\title{A Simple Example}
\author{Ali Aburas}

\begin{document}
\maketitle

\begin{abstract}
	Welcome to CS361-001-W2017\footnote{Software Engineering I}
\end{abstract}


\tableofcontents
%\newpage

\section{Motivation}
This is content in the top level section.
\subsection{A subsection heading.}
But content can be here.
\subsubsection{It's turtles all the way down!}
Or here.
\section{Approach}
What is your high-level approach?

\section{Something Important}
Two tools you should know about are:
%List LATEX provides a number of environments for lists of items:
%itemize For itemized (bulleted) lists.
%enumerate For numbered lists.
%description For tagged “description” lists.
\begin{description}
	\item[Version control]
		We require that you use a distributed version control system, such as \href{https://git-scm.com/}{GIT}.
	\item[LaTeX]
		We require that you use \LaTeX\ for the documentation.
	\end{description}


%Tables and Figures
\section{Figures and Tables}
\begin{figure}[h]
	\lstinputlisting {hello.c}
	\caption{This is a source code listing.}
	\label{fig:hello.c}
\end{figure}

And you can refer to figures directly by label (see Figure \ref{fig:hello.c}).

\begin{table}[h]
	\begin{tabular}{c|c|cc}
		This & and & another & thing\\
		Could & all & be & tabular\\
	\end{tabular}
\end{table}


\end{document}

